
\section{\joins\ Library Reference}
\label{joinsref}

Programs that use the \joins\ library should reference the assembly
\[ \texttt{Microsoft.Research.Joins.dll} \]
located in the \texttt{Assemblies} directory of the installation and, optionally, use the namespace:
\[ \texttt{using Microsoft.Research.Joins;} \]

All of the types described here reside in the \texttt{Microsoft.Research.Joins} namespace.


% The \cjoin\ class provides a mostly declarative, type-safe mechanism for
% defining thread-safe asynchronous and synchronous
% communication channels and patterns, for use by
% concurrently executing local threads or remote processes.
% The communication "channels" are special delegate values
% obtained from a common join object. Communication and/or
% synchronization takes place by invoking these delegates,
% passing arguments and optionally waiting for replied
% return values.  The allowable communication patterns as
% well as their effects are declared using join patterns:
% bodies of code whose execution is guarded by linear
% combinations of channels. The body, or continuation, of a
% join pattern is provided by the user as a delegate that
% may manipulate external resources protected by the join
% object.

% Here are two examples:

% \begin{verbatim}
% /* An unbounded, unordered string buffer.
%    Producers send strings by calling b.Put(s), never waiting. 
%    Consumers receive strings by calling b.Get(), possibly waiting 
%    until/unless there is a matching b.Put(s). */
% class Buffer {
%  /* Declare the (a)synchronous channels */
%  public readonly Asynchronous.Channel<string> Put;
%  public readonly Synchronous<string>.Channel Get;
%  public Buffer() {
%    /* Allocate a new Join object for this buffer */
%    Join join = Join.Create();
%    /* Use it to initialize the channels */
%    join.Initialize(out Put);  
%    join.Initialize(out Get);
%    /* Finally, declare the join patterns(s) */
%    join.When(Get).And(Put).Do(delegate(string s) { /* the continuation */
%       return s; 
%    });
%  }
% }
% \end{verbatim}

% \begin{verbatim}
% /* An object of type WaitForN<R> waits for n replies of type R, asynchronously.
%    The example demonstrates joining with an array of asynchronous channels.
%    Producer i posts his results on w.Responses[i], asynchronously.
%    The consumer calls w.Wait(), blocking until/unless all responses have come in. */
% public class WaitForN<R> { 
%  public readonly Asynchronous.Channel<R>[] Responses;
%  public readonly Synchronous<R[]>.Channel Wait;
%  public WaitForN(int n) {
%      Join j = Join.Create(n + 1);
%      j.Initialize(out Responses, n);
%      j.Initialize(out Wait);
%      j.When(Wait).And(Responses).Do(delegate(R[] results) { /* the continuation */
%        return results; });
%  }
% }
% \end{verbatim}


A new \cjoin\ instance \joinobj\ is allocated by calling an overload of factory method \methodref{Join.Create}. 

\[ 
\begin{array}{lr}
  \cjoin\; \joinobj = \joindotcreate(); & \mbox{or} \\ 
  \cjoin\; \joinobj = \joindotcreate(\textit{size});\\
\end{array}
\]
\noindent The second overload takes an integer $\textit{size}$ argument. It is used to explicitly bound the number of channels supported by 
\cjoin\ instance \joinobj. An omitted \textit{size}\ argument defaults to 32. 
The \textit{size} argument provides the value for the constant, readonly property $\joinobj\dotsize$.

A \cjoin\ object notionally owns a set of asynchronous and synchronous \emph{channels}, each obtained by calling an overload of
method \Initialize, passing the location, \textit{channel}(\textit{s}), of a channel or array of channels using an \out\ argument:\footnote{Languages that do not support \texttt{out} parameters can use alternative methods
\texttt{CreateChannel()} and \texttt{CreateChannels()} provided by classes \texttt{Asynchronous}, \texttt{Synchronous},
 $\texttt{Synchronous<\tvarr>}$.}
\[ 
\begin{array}{l}
  \joinobj\dotinitialize(\out\ \textit{channel});\\
  \joinobj\dotinitialize(\out\ \textit{channels}, \textit{length}); 
\end{array}
\]
\noindent The second form takes an integer $\textit{length}$ argument and initializes the location \textit{channels} with an 
an \emph{array} of \textit{length} distinct, asynchronous channels.



\emph{Asynchronous} channels are instances of these (nested) delegate types: % \dasync\ or \dasyncA{\tvara}:
\[ 
\begin{array}{l}
  \texttt{public delegate void } \dasync();\\
  \texttt{public delegate void } \dasyncA{\tvara}(\tvara\ a);
\end{array}
\]
\emph{Synchronous} channels are instances of  these (nested) delegate types: % \dsyncRA\tvarr\tvara,\dsyncR\tvarr, \dsyncvoidA\tvara, or \dsyncvoid.
\[ 
\begin{array}{l}
  \texttt{public delegate } \tvarr\; \dsyncRA{\tvarr}{\tvara}(\tvara\;a);\\
  \texttt{public delegate } \tvarr\; \dsyncR{\tvarr}();\\
  \texttt{public delegate void } \dsyncvoidA{\tvara}(\tvara\;a);\\
  \texttt{public delegate void } \dsyncvoid();\\
\end{array}
\]

Notice that the outer class of a channel, \texttt{Asynchronous}, \texttt{Synchronous} or $\texttt{Synchronous<}$\tvarr$\texttt{>}$,
should be read as a modifier that specifies its blocking behaviour and optional return type.

The various channel flavours support zero or one arguments of type \texttt\tvara, and zero or one results of type \texttt\tvarr.
If required, multiple arguments or results can be passed using user-declared tuple types or the provided generic
$\cpair{\tvara}{\tvarb}$ type.

When a synchronous channel is invoked, the caller
is blocked until the delegate returns (void or some value). 
When an asynchronous channel is invoked, there is no result and the
caller proceeds immediately without being blocked.

Apart from its channels, a \cjoin\ object notionally owns a set of \emph{join patterns}. Each pattern is defined by invoking an overload of 
the instance method \When\, followed by zero or more invocations of instance method \And\, followed by 
a final invocation of instance method \Do. Thus a pattern definition typically takes the form:
\[
  \joinobj\dotwhen(\channel_1)\dotand(\channel_2) \cdots \dotand(\channel_n)\dotdo(\delegate);
\]
Alternatively, using an anonymous delegate for \delegate: 
\[
  \joinobj\dotwhen(\channel_1)\dotand(\channel_2) \cdots \dotand(\channel_n)\dotdo(\delegatekw(P_1\;p_1, \ldots, P_m\;p_m)\{\ldots\});
\]

Here, the initial argument $\channel_1$ to $\When(\channel_1)$ may be a synchronous or asynchronous channel or an array of asynchronous channels.
Each subsequent argument  $\channel_i$ to $\And(\channel_i)$ (for $i>1$)  must be an asynchronous channel or an array of asynchronous channels; 
it cannot be a synchronous channel. The argument \delegate\ to $\Do(\delegate)$ is a \emph{continuation} delegate that defines the body of the pattern.
Although it varies with the pattern,  the type of the continuation is always an instance of one of the following delegate types:\footnote{
These are actually nested delegate types:
\[ 
\begin{array}{l}
  \texttt{public delegate } \tvarr\; \dlongreturn{\tvarr}{P_1,\ldots,P_m}(P_1\;p_1, \ldots,  P_m\;p_m );\\
  \texttt{public delegate void } \dlongreturnvoid{P_1,\ldots,P_m}(P_1\;p_1, \ldots,  P_m\;p_m );\\
\end{array}
\]
so their apparently free type parameters are bound by an enclosing class.}

\[ 
\begin{array}{l}
  \texttt{public delegate } \tvarr\; \dreturn{\tvarr}{P_1,\ldots,P_m}(P_1\;p_1, \ldots,  P_m\;p_m );\\
  \texttt{public delegate void } \dreturnvoid{P_1,\ldots,P_m}(P_1\;p_1, \ldots,  P_m\;p_m );\\
\end{array}
\]

The precise type of the continuation \delegate, including its number of arguments, is determined by the sequence of channels guarding it.
If the first argument $\channel_1$ in the pattern is a synchronous  channel with return type $\tvarr$, then
the continuation's return type is $\tvarr$; otherwise the return type is $\texttt{void}$.


% The pattern's body \delegate\ is a 
% (returning an \texttt\tvarr), or $\dreturnvoid{P_1,\ldots,P_m}$ (returning  
% \texttt{void}) depending on whether the head $\channel_1$ is a synchronous channel that returns a result of some type \texttt\tvarr.
The continuation receives the arguments of the joined channel invocations 
as delegate parameters $P_1\;p_1, \ldots,  P_m\;p_m$, for $m\leq n$.
The presence and types of any additional parameters $P_1\;p_1, \ldots,  P_m\;p_m$
varies according the type of each argument $\channel_i$ joined with 
invocation $\When(\channel_i)/\And(\channel_i)$ (for $1\leq i\leq n$):

\begin{itemize}
\item
If $\channel_i$ is of type $\dsyncR{\tvarr}$, $\dsyncvoid$, $\dasync$ or  $\dasync[]$, then $\When(\channel_i)/\And(\channel_i)$ adds no parameter to delegate \delegate.

\item
If $\channel_i$ is of type $\dsyncRA{R}{P}$, $\dsyncvoidA{P}$, \\ or $\dasyncA{P}$  for some type $P$ then $\When(\channel_i)/\And(\channel_i)$ 
adds one parameter $p_j$ of type $P_j=P$ to delegate \delegate.

\item
If $\channel_i$ is an array of type $\dasyncA{P}[]$ for some type $P$ then $\When(\channel_i)/\And(\channel_i)$ 
adds one parameter $p_j$ of array type $P_j=P[]$ to delegate \delegate.
\end{itemize}

Parameters are added to \delegate\ from left to right, in increasing order of 
$i$. In the current implementation, a continuation can receive at most $m\leq\maxbindings$ parameters. If necessary,
 it is possible to join more than \maxbindings\ generic channels by calling  method $\AndPair(\channel_i)$ instead of $\And(\channel_i)$. 
$\And(\channel_i)$  modifies the last argument of the new continuation to be a pair consisting of the last argument
of the previous continuation and the new argument contributed by $ai$.




A join pattern associates a set of (a)synchronous channels with a body \delegate.
The body of a pattern can only execute once all the
channels guarding it have been invoked. Thus, when a channel is 
invoked there may be zero, one, or more patterns which are enabled:
\begin{itemize}
\item
If no pattern is enabled then the channel invocation is queued up. If the channel is asynchronous,
then this simply involves adding the arguments (the contents of the message) to an internal queue.
If the channel is synchronous, then the calling thread is blocked, joining a notional queue of threads waiting on this channel.
\item
If there is a single enabled join pattern, then the arguments of the calls involved in the match are de-queued,
any blocked thread involved in the match is awakened, and the body of the pattern is executed in one of the threads.
\item
When a join pattern that involves only asynchronous channels is enabled, then it runs in a new thread.
If there are several enabled patterns, then an unspecified one is chosen to run.
\item
Similarly, if there are multiple invocations of a particular (a)synchronous channel queued up, 
which call will be de-queued when there is a match is unspecified.
\end{itemize}

The current number of channels initialized on $\joinobj$ is available as readonly property $\joinobj\dotcount$; 
its value is bounded by $\joinobj\dotsize$.
Any invocation of $\joinobj\dotinitialize$ that would cause $\joinobj\dotcount$ to exceed $\joinobj\dotsize$ throws \typeref{JoinException}.


Join patterns must be well-formed, both individually and collectively. Executing $\Do(\delegate)$ to
complete a join pattern will throw \typeref{JoinException} if $\delegate$ is
\texttt{null}, the pattern repeats an asynchronous channel, an (a)synchronous
channel is \texttt{null} or \emph{foreign} to this pattern's \cjoin\ instance, the
join pattern is \emph{redundant}, or the join pattern is \emph{empty}.
A channel is foreign to a \cjoin\ instance \joinobj\ if it was not allocated by some call to
$\joinobj\dotinitialize$.
A pattern is redundant when the set
of channels joined by the pattern subsets or supersets the channels
joined by an existing pattern on this \texttt{Join} instance.
A pattern is empty when the set of channels joined by the pattern is the empty set 
(emptiness can only arise from joining empty arrays of channels).






