\documentclass{article}
\newcommand{\comega}{\mbox{C$\omega$}}
\newcommand{\csharp}{\mbox{C$^\sharp$}}
\title{\comega\ Concurrency Constructs}
\author{Nick Benton}
\bibliographystyle{plain}
\begin{document}
\maketitle
\section{Getting Started with \comega\ Concurrency}

\subsection{Overview}
\comega\ extends the \csharp\ programming language with new
asynchronous concurrency abstractions. The language presents a simple
and powerful model of concurrency which is applicable both to
multithreaded applications running on a single machine and to the
orchestration of asynchronous, event-based applications communicating
over a wide area network.

The new constructs are a mild syntactic variant of those we have
previously described under the name `Polyphonic \csharp'
\cite{polyphony:ecoop} -- \comega\ combines Polyphonic \csharp\ with
the rich new data programming model described elsewhere.


\subsection{The Basic Idea}
In \comega, methods can be defined as either \emph{synchronous} or
\emph{asynchronous}. When a synchronous method is called, the caller
is blocked until the method returns, as is normal in \csharp. However,
when an asynchronous method is called, there is no result and the
caller proceeds immediately without being blocked. Thus from the
caller's point of view, an asynchronous method is like a \verb|void| one, but
with the useful extra guarantee of returning immediately. We often
refer to asynchronous methods as \emph{messages}, as they are a
one-way communication from caller to receiver (think of posting a
letter rather as opposed to asking a question and waiting for an
answer during a face-to-face conversation).

By themselves, asynchronous method declarations are not particularly
novel. Indeed, .NET already has a widely-used set of library classes
which allow any method to be invoked asynchronously (though note that
in this standard pattern it is the caller who decides to invoke a
method asynchronously, whereas in \comega\ it is the callee
(defining) side which declares a particular method to be
asynchronous). The significant innovation in \comega\ is the way
in which method bodies are defined.

In most languages, including \csharp, methods in the signature of a
class are in bijective correspondence with the code of their
implementations -- for each method which is declared, there is a
single, distinct definition of what happens when that method is
called. In \comega, however, a body may be associated with a \emph{set}
of (synchronous and/or asynchronous) methods. We call such a
definition a \emph{chord}, and a particular method may appear in the header
of several chords. The body of a chord can only execute once \emph{all} the
methods in its header have been called. Thus, when a polyphonic method
is called there may be zero, one, or more chords which are enabled:
\begin{itemize}
\item If no chord is enabled then the method invocation is queued
up. If the method is asynchronous, then this simply involves adding
the arguments (the contents of the message) to a queue. If the method
is synchronous, then the calling thread is blocked.
\item If there is a single enabled chord, then the arguments of the
calls involved in the match are de-queued, any blocked thread involved
in the match is awakened, and the body runs.
\item When a chord which involves only asynchronous methods runs, then
it does so in a new thread.
\item If there are several chords which are enabled then an
unspecified one of them is chosen to run.
\item Similarly, if there are multiple calls to a particular method
queued up, we do not specify which call will be de-queued when there
is a match.
\end{itemize}


\subsubsection{Example: A Simple Buffer}

Here is the simplest interesting example of a \comega\ class:
\begin{verbatim}
public class Buffer {
   public async Put(string s);
   
   public string Get() & Put(string s) { 
      return s; 
   } 
} 
\end{verbatim}
This class contains two methods: a synchronous one, \verb|Get()|,
which takes no arguments and returns a string, and an asynchronous
one, \verb|Put()|, which takes a string argument and (like all
asynchronous methods) returns no result. The class definition contains
two things: a declaration (with no body) for the asynchronous method,
and a chord. The chord declares the synchronous method and defines a
body (the return statement) which can run when \emph{both} the
\verb|Get()| and \verb|Put()| methods have been called.

Now assume that producer and consumer threads wish to communicate via
an instance \verb|b| of the class \verb|Buffer|. Producers make calls
to \verb|b.Put()|, which, since the method is asynchronous, never
block. Consumers make calls to \verb|b.Get()|, which, since the method
is synchronous, will block until or unless there is a matching call to
\verb|Put()|. Once \verb|b| has received both a \verb|Put()| and a
\verb|get()|, the body runs and the argument to the \verb|Put()| is
returned as the result of the call to \verb|Get()|. Multiple calls to
\verb|Get()| may be pending before a \verb|Put()| is received to
reawaken one of them, and multiple calls to \verb|Put()| may be made
before their arguments are consumed by subsequent \verb|Get()|s. Note
that
\begin{enumerate}
\item The body of the chord runs in the (reawakened) thread
corresponding to the matched call to \verb|Get()|. Hence no new threads are
spawned in this example.
\item The code which is generated by the class definition above is
completely thread safe. The compiler automatically generates the
locking necessary to ensure that, for example, the argument to a
particular call of \verb|Put()| cannot be delivered to two distinct calls to
\verb|Get()|. Furthermore (though it makes little difference in this small
example), the locking is fine-grained and brief - polyphonic methods
to not lock the whole object and are not executed with ``monitor
semantics''.
\item The reader may wonder how we know which of the methods involved
in a chord gets the returned value. The answer is that it is always
the synchronous one, and there can be at most one synchronous method
involved in a chord.
\end{enumerate}

In general, the definition of a synchronous method in \comega\
consists of more than one chord, each of which defines a body which
can run when the method has been called \emph{and} a particular set of
asynchronous messages are present. For example, we could modify the
example above to allow \verb|Get()| to synchronize with calls to
either of two different \verb|Put()| methods:
\begin{verbatim}
public class Buffer {
   public async Put(string s);
   public async Put(int n);
   
   public string Get() 
   & Put(string s) { 
      return s; 
   } 
   & Put(int n) {
      return n.ToString();
   }
   
} 
\end{verbatim}
Now we have two asynchronous methods (which happen to be
distinguished by type rather than name) and a synchronous method which
can synchronize with either one, with a different body in each case.

\subsubsection{Example: A One-place Buffer}
The previous example showed how to define a buffer of unbounded size:
any number of calls to \verb|Put()| could be queued up before matching a
\verb|Get()|. We now define a variant in which only a single data value may
held in the buffer at any one time:
\begin{verbatim}
public class OnePlaceBuffer {
   private async empty();
   private async contains(string s);

   public OnePlaceBuffer() {
      empty();
   }

   public void Put(string s) & empty() {
      contains(s);
      return;
   }

   public string Get() & contains(string s) {
      empty();
      return s;
   }
}
\end{verbatim}
The public interface of OnePlaceBuffer is similar to that of Buffer,
although the \verb|Put()| method is now synchronous, since it can now block
in the case that there is already an unconsumed value in the buffer.

The implementation of OnePlaceBuffer makes use of two private
asynchronous messages: \verb|empty()| and \verb|contains()|. These are
used to carry the state of the buffer and illustrate a very common
programming pattern in \comega: note that we have made no use of
fields. The way in which this works can be best understood by reading
the constructor and the two chords in a simple declarative manner:
\begin{itemize}
\item When a new buffer is created, it is initially \verb|empty()|. 
\item If you call \verb|Put(s)| on an \verb|empty()| buffer then it subsequently \verb|contains(s)| and the call to \verb|Put()| returns. 
\item If you call \verb|Get()| on a buffer which \verb|contains(s)| then the buffer is subsequently \verb|empty()| and \verb|s| is returned to the caller of \verb|Get()|. 
\item Implicitly: In all other cases, calls to \verb|Put()| and \verb|Get()| block. 
\end{itemize}
It is easy to see that the constructor establishes, and both the
chords maintain, the invariant that there is always exactly one
\verb|empty()| or \verb|contains(s)| message pending on the
buffer. Thus the two chords can be read as a fairly direct
specification of a finite state machine.

\section{\comega\ Concurrency Tutorials}
\subsection{Spawning Threads}
\comega\ is built on the existing \csharp /.NET implementation of
threads. Thus one can start new threads in a \comega\ program by the
usual technique of calling \verb|Start()| on an instance of
\texttt{System.Threading.Thread} which was constructed from an
instance of the delegate
\texttt{System.Threading.ThreadStart}. However, the delegate must be
created from a method with no arguments, so passing parameters to
threads created this way requires some extra work (see the Creating
Threads topic in the .NET documentation). In \comega, the body of a
chord which contains \emph{only} asynchronous methods will execute in
a new thread. In particular, calling an asynchronous method which is
defined in a chord by itself gives an easy way to start a new thread
with parameters. For example:
\begin{verbatim}
using System;
using System.Threading;

public class SpawningTutorial {
    static async m(string s);

    static void Main() {
        m("thread one");
        m("thread two");
    }
    
    when m(string s) {
      while(true) {
        Console.WriteLine(s);
        Thread.Sleep(1000);
      }
    }
}
\end{verbatim}
prints
\begin{verbatim}
thread one
thread two
thread one
thread two
thread one
thread two
...
\end{verbatim}
Whenever the method \verb|m()| is called, the call returns immediately
and the body is executed in a new thread. Note that chords which
contain only asynchronous methods are introduced with the keyword
\verb|when|.


\subsection{Using Asynchronous Messages For State}
Objects which have mutable state and which may be accessed from
multiple threads usually have to ensure that methods which read or
write that state are executed with some form of mutual exclusion. In \csharp, the simplest way of achieving this is to use the \verb|lock| statement\footnote{or a \texttt{System.Runtime.CompilerServices.MethodImplAttribute} or explicit calls to \texttt{System.Threading.Monitor.Enter}/\texttt{Exit}} as in
\begin{verbatim}
public class Counter {
 long i;

 public Counter {
   i=0;
 }

 public void Inc {
   lock (this) {        // acquire the lock on this object
     i = i+1;           // update the state
   }                    // release the lock
 }

 public long Value() {
   lock (this) {
     return i;
   }
 }
}
\end{verbatim}
Of course, the above still works in \comega, but there are also alternatives. Here is one:
\begin{verbatim}
public class Counter {
  long i;
  async mylock();

  public Counter {
    i=0;
    mylock();
  }

  public void Inc() & mylock() { // acquire the lock
    i = i+1;                     // update the state
    mylock();                    // release the lock
  }

  public long Value() & mylock() {
    long r = i;
    mylock();
    return r;
  }
}
\end{verbatim}
In this case we have used a private asynchronous message
\verb|mylock()| to protect access to the field \verb|i|. When the
\verb|Inc()| method is called, it will block until/unless there is a
waiting \verb|mylock|. When there is, then the call proceeds and the
\verb|mylock()| message is consumed, so further calls to \verb|Inc()|
or \verb|Value()| from other threads will block until another
\verb|mylock()| is sent. Thus the thread which called \verb|Inc()| has
exclusive access to the field whilst it does the increment. Once it
has finished, it sends another \verb|mylock()| to allow a subsequent
call access. Similar reasoning applies to \verb|Value()|. Observe that
there is always at most one call to \verb|mylock()| pending on a
counter.

Going a step further, we can eliminate the field entirely. Instead we
pass the state of the counter around as an argument to a private
asynchronous message:
\begin{verbatim}
public class Counter {
  async mystate(long i);

  public Counter {
    mystate(0);
  }

  public void Inc() & mystate(long i) { // acquire the state
    mystate(i+1);
  }

  public long Value() & mystate(long i) {
    mystate(i);
    return i;
  }
}
\end{verbatim}
In such a trivial case, using \comega\ constucts to implement mutual
exclusion this way is probably a bad idea - it is much the same length
as the version which uses \verb|lock|, is less efficient and has more
opportunities for error (a typical example is forgetting to resend the
state-carrying message at the end of a method). However, in cases
where one wants more fine-grained locking on parts of the state of an
object, or to block and awaken particular threads when certain
conditions are established (i.e. situations in which the \csharp\
programmer would use \verb|Notify()| and \verb|Pulse()|), this style
of programming becomes preferable.

\subsubsection{Example: A Reader-Writer Lock}

Controlling concurrent access to a shared, mutable resource is a
classic problem. Clients request, and later release, either read-only
or read-write access to the resource. To preserve consistency but
minimize waiting, one usually wishes to allow either any number of
readers or a single writer, but not both, to have access at any one
time. The code below implements a reader-writer lock in
\comega. Writers call \verb|Exclusive()|, do some writing and then
call \verb|ReleaseExclusive()|. Readers call \verb|Shared()|, do some
reading and then call \verb|ReleaseShared()|:
\begin{verbatim}
public class ReaderWriter {
   private async idle();
   private async s(int n);

   public ReaderWriter() {
      idle();
   }

   public void Exclusive() 
   & idle() {
   }

   public void ReleaseExclusive() {
      idle(); 
   } 

   public void Shared() 
   & idle() {
      s(1);
   }
   & s(int n) {
      s(n+1);
   }

   public void ReleaseShared() 
   & s(int n) { 
      if (n == 1) 
        idle(); 
      else 
        s(n-1); 
   }
}
\end{verbatim}

Again, we use private messages, \verb|idle()| and \verb|s(n)|, to carry the state. And once more, there is a simple declarative reading  of the constructor and chords which allows one to understand the implementation:
\begin{itemize}
\item When the lock is created there are no readers or writers, so it
is \verb|idle()|.
\item If a writer requests \verb|Exclusive()| access and the lock is
\verb|idle()| then he may proceed (but the lock is no longer
\verb|idle()|).
\item If a writer indicates he is finished by calling
\verb|ReleaseExclusive()| then the lock is \verb|idle()| again.
\item If a reader requests \verb|Shared()| access and the lock is
\verb|idle()| then the lock moves to the state \verb|s(1)|, meaning
there is now one reader, and he may proceed.
\item If a reader requests \verb|Shared()| access and there are
currently $n$ readers, then there are now $n+1$ readers, and the new
one may proceed.
\item If a reader indicates he is finished by calling
\verb|ReleaseShared()| and there were previously $n$ readers, then if $n$
was $1$ then lock is \verb|idle()| again, otherwise there are now $n-1$ shared
readers. In either case, the reader who has just relinquished access
may proceed with whatever else he has to do.
\end{itemize}

Assuming that all the clients obey the request-release protocol, the
invariant is that there is at most one pending private async message
representing the state:
\[
      none \leftrightarrow idle() \leftrightarrow s(1) 
\leftrightarrow s(2) \leftrightarrow s(3) \leftrightarrow\ldots
\]
Operationally, it may help to think about what does \emph{not} appear in the
code. There is, for example, no chord which is applicable if a client
calls \verb|Exclusive()| when there is a pending \verb|s(n)| message. In such a
case, the client will block until all the readers have released their
access and an \verb|idle()| message has been sent.\footnote{There is a question about
fairness here - see the Polyphonic \csharp\ paper for a modification
to this example which enforces a simple form of fairness between
readers and writers.}

\subsubsection{Example: A Bounded Buffer}
The `Getting Started' section gives examples of an unbounded buffer
(for which producers are never blocked) and a single-slot buffer (at
most one value may be held in the buffer). Here we show how one might
program a general $n$-place buffer in \comega. There are several
approaches one could take, but the following seems the neatest:
\begin{verbatim}
public class BufferN {
  private async token();
  private async value(object o);

  public BufferN(int size) {
    for(int i=0;i<size;i++) {
       token();
    }
  }

  public void Put(object o) & token() {
    value(o);
  }

  public object Get() & value(o) {
    token();
    return o;
  }
}
\end{verbatim}
When an $n$-slot buffer object is created, it sends itself $n$
\verb|token()| messages. Calls to \verb|Put()| block until there is a
\verb|token()| available, consume it and send a \verb|value()| message
containing the object which was put in the buffer. Calls to
\verb|Get()| block until there is a \verb|value()| message available,
consume it and produce a fresh \verb|token()|. The invariant here, of
course, is that the number of \verb|token()| messages plus the number
of \verb|value()| message always equals the size of the buffer.

\subsection{Asynchronous Communication}
The previous section describes how private asynchronous messages may
be used to carry state. Here we show how to orchestrate asynchronous
messages between different objects.

\subsubsection{Semaphores}
A semaphore exposes two methods, \verb|Wait()|, which is synchronous, and \verb|Signal()|, which is asynchronous. Calls to \verb|Wait()| block until there has been a matching call to \verb|Signal()|:
\begin{verbatim}
public class Semaphore {
  public async Signal();

  public void Wait() & Signal() {
  }
}
\end{verbatim}
For example, here is one way to structure a program which spawns two
new threads and then waits for them both to finish:
\begin{verbatim}
public class MyApp {
  static async task1(int arg, Semaphore s);
  static async task2(int arg, Semaphore s);

  public static void Main() {
    Semaphore s = new Semaphore();

    task1(1,s);
    task2(2,s);
    // do something in main thread
    s.Wait(); // wait for one to finish
    s.Wait(); // wait for another to finish
    Console.WriteLine("both tasks finished");
  }

  when task1(int arg, Semaphore s) {
     // do something 
     s.Signal(); // say we've finished
  }
  when task2(int arg, Semaphore s) {
     // do something else
     s.Signal(); // say we've finished
  }
}
\end{verbatim}

\subsubsection{Asynchronous call and return patterns}
Calling an asynchronous method is a one-way fire and forget
operation. How then does one get results back from asynchronous calls?
Firstly, one must explicitly tell the receiver of the call what do to
with the result. This is often done by passing an extra parameter
along with the arguments to the call. This extra argument can be of
some application specific agreed type (for example, a buffer into
which the result is to be put or the name of a file into which the result is
to be written) or, more generally, a delegate (callback) which should
be invoked with the result. In terms of the analogy between calling an
asynchronous method and sending a letter, the extra parameter
corresponds to the ``return address'' which one adds to the letter's
actual content. Thus the interface to a typical service which can be
invoked asynchronously and which returns a result asynchronously might
look like this:
\begin{verbatim}
public delegate async MyCallback(object result);

public interface IService {
  public async Service(object parameter, MyCallback cb);
}
\end{verbatim}
where the implementation of \texttt{Service} will do some computation
with the parameter and then invoke the callback with a result.

The second interesting aspect of asynchronous programming is that
messages arrive at unpredictable times and in unpredictable
orders. Amongst other things, this means that correlating requests and
responses can require some extra programming, as it may not otherwise
be clear what a given response refers to. The correlation
information can take the form of extra explicit tokens passed to,
and returned back from, a service (corresponding to ``your ref:
whatever'' or ``please quote order number nnn in future
correspondence'' in a letter). Alternatively, the correlation tokens
can be passed more implicitly by creating new target objects for the
callbacks sent with different requests (this is a bit like adding
``Dept. XYZnn'' to the return address of a letter). One also often
wishes to coordinate multiple asynchronous messages, for example
performing an action once a number of replies have been received or a
certain time has passed. For example, here's how one might invoke two
asynchronous services and subsequently wait for replies from both of
them:
\begin{verbatim}
  class Join2 {
    public MyCallback firstcb; 
    public MyCallback secondcb;
    async first(object Fst);
    async second(object Snd);

    public Join2() {
      firstcb  = new MyCallback(first);
      secondcb = new MyCallback(second);
    }

    public void Wait(out object x, out object y) 
    & first(object Fst)
    & second(object Snd) {
      x = Fst; y = Snd;
    }
  }

  class Client {
    public static void Main(string[] args) {
      IService s1 = ... ;
      IService s2 = ... ;
      Join2 j = new Join2();
      s1.Service(args[0], j.firstcb);
      s2.Service(args[1], j.secondcb);
      ...                   // do something useful in the meantime...
      object x,y;
      j.Wait(out x, out y); // wait for both results to come back
      ...                   // do something with them
    }
  }
\end{verbatim}
\verb|Join2| is a library class which encapsulates the pattern of waiting for
two responses. The client creates an instance of \verb|Join2|, passes its
callbacks along with the arguments to two services and then carries on
with its own processing. Later, it calls \verb|Wait()| which will block
until/unless both of the callbacks have been invoked. Note that the
correlation here is of the `target address' (demultiplexed) form,
rather than the `explicit token' (multiplexed) form; this is usually
the preferred style in \comega, since synchronization behaviour cannot
depend directly on message contents.

Writing classes like \verb|Join2| which wait for just one result
(Select) or more complex conditions (e.g. at least $m$ out of $n$) is
straightforward.

\subsubsection{Active Objects}
One common pattern of asynchronous programming is that of active
objects, or actors. Active objects communicate by asynchronous
messages which are processed sequentially by a thread of control which
is specific to that object. One way of programming active objects in
\comega\ is by inheritance from an abstract base class:
\begin{verbatim}
public abstract class ActiveObject : MarshalByRefObject {
  protected bool done = false;
  async start();
  protected abstract void ProcessMessage();
  
  public ActiveObject() {
    start();
  }
	   
  when start() {
    while (!done) {
      ProcessMessage();
    }
  }
}
\end{verbatim}
When an instance of \texttt{ActiveObject} is created, the \texttt{start()}
method is called, which spawns a new thread which repeatedly calls the
abstract synchronous method \texttt{ProcessMessage()}. Subclasses of
\texttt{ActiveObject} then override \texttt{ProcessMessage()}, joining it
in a sequence of chords with each of the asynchronous messages which
they wish to process. For example, a message redistributor object in
an event-based application might look like:
\begin{verbatim}
public interface EventSink {
  async Post(string message);
}

public class Distributor : ActiveObject, EventSink {
  private ArrayList subscribers = new ArrayList();
  private string myname;
  
  public async Subscribe(EventSink sink);
  public async Post(string message);
  
  protected override void ProcessMessage() 
  & Subscribe(EventSink sink) {
    subscribers.Add(sink);
  } 
  & Post(string message) {
    foreach (EventSink sink in subscribers) {
      sink.Post(myname + ":" + message);
    }
  }
  
  public Distributor(string name) {
    myname = name;
  }
}
\end{verbatim}
A \verb|Distributor| object can receive asynchronous \verb|Subscribe| and
\verb|Post| messages. Each \verb|Post| message is resent to
each of the current subscribers. No locking is required for access to
the current subscriber list since, although messages can arrive at any
time, they are processed strictly sequentially by the \verb|Distributor|'s
message-loop thread (without blocking their sender, of course).

Since threads are a relatively expensive resource on the CLR, having
many instances of this \verb|ActiveObject| class around at once will
be inefficient. However, ActiveObjects are not built in to \comega\ -
they're just one example of the sort of thing one can build easily
with the \comega\ concurrency constructs. Implementing variants of
this pattern, such as families of objects which share a custom
threadpool implementation is also straightforward.


\section{\comega\ Concurrency Constructs Reference}
\subsection{Syntax}
This section gives a nearly-formal specification of the extensions to the
\csharp\ grammar associated with \comega\ concurrency constructs.
\subsubsection{Asynchronous methods}
Asynchronous methods have a return type of \texttt{async}, which is a
new keyword in \comega\ (similar to \texttt{void}). 
Each asynchronous method in a class must
have an explicit \emph{async-declaration} of the form
\[
\textit{attributes}_{opt}\ \textit{method-modifiers}_{opt}\
\texttt{async}\ \textit{member-name}\ \texttt{(}\
\textit{formal-parameter-list}_{opt} \texttt{);}
\]
Asynchronous.Channel-declarations can appear in interfaces and one can also declare
asynchronous delegate types:
\[
\textit{attributes}_{opt}\ \textit{delegate-modifiers}_{opt}\
\texttt{delegate async}\ \textit{identifier}\ \texttt{(}\
\textit{formal-parameter-list}_{opt} \texttt{);}
\]
\subsubsection{Chords}
Chords (or synchronization patterns) define the behaviours of
asynchronous methods and of those (ordinary)
synchronous methods which synchronize with asynchronous ones. The
definition \emph{method-declaration} in the \csharp\ grammar is
replaced by the following in \comega:
\[
\textit{sync-method-header}\ \textit{sync-chord}^{+}
\]
where \emph{sync-method-header} is
\[
\textit{attributes}_{opt}\ \textit{method-modifiers}_{opt}\
(\texttt{void} | \textit{type})\ \textit{member-name}\ \texttt{(}\
\textit{formal-parameter-list}_{opt} \texttt{)}
\]
and \emph{sync-chord} is
\[
(\texttt{\&}\ \textit{async-header})^\ast\ \textit{method-body}
\]
where \emph{async-header} is
\[
\textit{member-name}\ \texttt{(}\
\textit{formal-parameter-list}_{opt} \texttt{)}
\]

There are also another form of \emph{class-member-declaration} in
\comega, viz. \emph{async-chord}, which is defined by
\[
\texttt{when}\ \textit{async-header}\ (\texttt{\&}\
\textit{async-header})^\ast\ \textit{method-body}
\]
where \texttt{when} is another new keyword.

\subsubsection{Well-formedness conditions}
\begin{itemize}
\item Asynchronous methods may not have \texttt{ref} or \texttt{out}
parameters.
\item Static and non-static methods may not be mixed within a chord.
\item The \emph{member-name} in any \emph{async-header} within a class
declaration must correspond to an \emph{async-declaration} of the same
name and parameter types within that class.
\item Although a given asynchronous method (identified by name and
parameter types) may occur in more than one \emph{sync-chord} and/or
\emph{async-chord}, all the asynchronous methods within a single chord
must be distinct. Within a single class, a particular synchronous
method may only occur in a single \emph{method-declaration}.
\item For each \emph{sync-chord} in a given \emph{method-declaration}, the formal parameters to the
synchronous method and the formal parameters in each
\emph{async-header} in the chord must all be distinct. Similarly, the
parameters in each \emph{async-header} in an \emph{async-chord} must
all be distinct.
\item The bound variables in each \emph{method-body} in a
\emph{method-declaration} are the formal parameters of the synchronous
method unioned with those of all the \emph{async-header}s in the
enclosing \emph{sync-chord}. Similarly, the bound variables of a
\emph{method-body} in an \emph{async-chord} are all the parameters of
the associated \emph{async-header}s.
\item The returned value (if any) of each \emph{method-body} in a
\emph{method-declaration} must match the declared return type of the
synchronous method being defined. Method bodies in \emph{async-chord}s
may not return a value.
\item In a \texttt{struct} definition, only static methods may occur in
non-trivial chords.
\item If a (synchronous or asynchronous) method declaration in a class
$C$ is marked \texttt{override} then any method which appears in a
chord with the \emph{overridden} method (in the superclass where that
is declared) must also be overridden in the class $C$.
\end{itemize}
This last restriction means that, for example, the following is
ill-formed:
\begin{verbatim}
public class C {
  virtual async g();

  virtual void f() & g() {
   ...
  }
  ...
}

public class D : C {
  override void f() {
    ...
  }
  ...
}
\end{verbatim}
Since in \texttt{D}, we have overridden the definition of \texttt{f()}
in \texttt{C} but have not also overridden \texttt{g()} with which it
appears in a chord. (If \texttt{g()} were non-virtual then the
definition of \texttt{C} alone would generate a warning, since
\texttt{f()} would be declared \texttt{virtual} but could never be
legally overridden.)

\subsection{Typing}
We treat \verb$async$ as a subtype of \verb$void$ and allow \emph{covariant
return types} just in the case of these two (pseudo)types. Thus
\begin{itemize}
\item an \verb$async$ method may override a \verb$void$
one, 
\item a \verb$void$ delegate may be created from
an \verb$async$ method, and
\item an \verb$async$ method may implement a \verb$void$ method in an interface
\end{itemize}
but not conversely. This design makes intuitive sense (an \verb$async$
method \emph{is} a \verb$void$ one, but has the extra property of
returning `immediately') and also maximizes compatibility with
existing {\csharp} code (superclasses, interfaces and delegate
definitions) making use of \verb$void$.
\subsection{Semantics}
An asynchronous method will return (essentially) immediately to its
caller without blocking and with no result. Asynchronous methods are
automatically given the \texttt{OneWay} attribute, so that calls to
them over remoting are genuinely non-blocking.

Each chord involves a non-empty set of methods. A chord is
\emph{enabled} on an object (or class, in the case of static chords)
when at least one \emph{unconsumed} call has been made to each of
those methods on that object (or class). When an asynchronous method
is called on an object (or class) and no chord is thereby enabled, the
call is queued. When a synchronous method is called on an object ()
and no chord is enabled, the calling thread is blocked and
queued. When at least one chord is enabled on an object (or class)
then (an unspecified) one of them will be chosen to fire, which
\emph{consumes} (dequeues) one (unspecified) call of each of the
methods in the chord and executes the body of the chord with the
formal parameters bound to the arguments of those calls. In the case
of a synchronous chord, the body is executed in the reawoken thread
which made the consumed synchronous call. In the case of an
asynchronous chord, the body is executed in a new thread.


\section{\comega\ Concurrency Samples}
These samples can all be found in the Samples subdirectory of the
\comega\ distribution.
\subsection{Dining Philosophers}
This sample implements an animated solution to the classic dining
philosophers problem. It demonstrates the use of asynchronous messages
to carry state (in the \texttt{Room} and \texttt{Fork} classes, for
example) as well as asynchronous messages to coordinate different
threads (the \texttt{endmove()} message, for example). There is one
thread for each philosopher and another thread (the GUI thread) which
runs the animation using a timer.

\subsection{Santa Claus}
This sample implements a solution to a concurrent programming problem
about Santa and his reindeer. For a full account of the problem and
the solution, please see the associated paper \cite{benton:santa}.

\subsection{TerraServer Client}
This sample implements a client to the TerraServer web service, which
provides USGS aerial imagery (see http://terraservice.net). It
requires a working connection to the internet (and the web service to
be running) in order to work.

Type a city and state name into the text boxes, click one of the
buttons and an aerial photograph will be displayed (this may take 30
or 40 seconds, even with a fast connection). TerraServer stores
imagery as small `tiles', several of which must be retrieved and
pasted together to make a whole picture. If you click `Synchronous'
then the individual tiles are requested and retrieved one at a time
from the server. If you click `Asynchronous' then asynchronous
requests for all the tiles are fired off together and the results are
then processed as they arrive. Making the requests asynchronously is
usually significantly faster, as shown by the time displays under each
button.

The sample illustrates `impedance matching' between the usual .NET
asynchronous calling pattern and {\comega}'s asynchronous methods. It
is also interesting because the reference to the web service was
created as a \csharp\ project within Visual Studio, which was then
linked with the \comega\ code to form the whole solution.

\subsection{Stock Trader}
This sample illustrates distributed programming using \comega\ with
.NET remoting. It comprises a command-line server application and a
GUI client application. Multiple clients (possibly running on
different machines) can connect to an instance of the server and make
bids and offers for stocks. The server matches up buyers and sellers
and notifies them of the outcome when there is a trade.

\bibliography{concurrency}
\end{document}